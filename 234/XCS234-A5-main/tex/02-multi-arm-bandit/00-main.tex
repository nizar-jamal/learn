\section{RL for personalized recommendations}

One of the most influential applications of RL in the real-world is in generating personalized recommendations for videos, movies, games, music, etc. Companies such as Netflix (\cite{netflix_slides}), Spotify (\cite{spotify_recs}), Yahoo (\cite{li2010contextual}) and Microsoft (\cite{swaminathan2017offpolicy}) use contextual bandits to recommend content that is more likely to catch the user's attention. Generating recommendations is an important task for these companies --- the value of Netflix recommendations to the company itself is estimated at \$1 billion (\cite{netflix_recs}).

\noindent
Content recommendations take place in a dynamical system containing feedback loops (\cite{Beer1995}) affecting both users and producers. Reading recommended articles and watching recommended videos changes people's interests and preferences, and content providers change what they make as they learn what their audience values.  However, when the system recommends content to the user, the user's choices are strongly influenced by the set of options offered. This creates a feedback loop in which training data reflects \textit{both} user preferences and previous algorithmic recommendations. 

\noindent
In this problem, we will investigate how \textit{video creators} learn from people's interactions with a recommendation system for videos, how they change the videos they produce accordingly, and what these provider-side feedback loops mean for the users. Dynamics similar to the ones we investigate here have been studied in newsrooms as journalists respond to real-time information about article popularity (\cite{Christin2018}), and on YouTube as video creators use metrics such as clicks, minutes watched, likes and dislikes, comments, and more to determine what video topics and formats their audience prefers (\cite{Christin2021}). 

\noindent
We have created a (toy) simulation that allows you to model a video recommender system as a contextual bandit problem.\footnote{According to \cite{45530}, YouTube does not currently use RL for their recommendations, but other video recommendation systems do, as noted above.} In our simulation, assume we have a certain fixed number of users $N_u$. Each user has a set of preferences, and their preference sets are all different from one another. We start off with some number of videos we can recommend to these users. These videos correspond to the arms of the contextual bandit. Initially there are $N_a$ arms. Your goal is to develop a contextual bandit algorithm that can recommend the best videos to each user as quickly as possible. 
 

\noindent
In our Warfarin setting above, $N_a$ was fixed: we always chose from three different dosages for all patients. However, video hosting and recommendation sites like YouTube are more dynamic. Content creators monitor user behavior and add new videos (i.e. arms) to the platform in response to the popularity of their past videos. In other words, $N_a$ keeps increasing over time. 

\noindent
How does this change the problem to be solved? Are we still in the bandit setting or is this now morphing into an RL problem? For now, we will treat it as a bandit problem. Remember that the number of users is static: $N_u$ is a constant and doesn't change. In the coding portion of this assignment, you will study the effect of adding new arms into the contextual bandit setup, the different strategies we can employ to add these arms and measure how they affect performance. 

\subsection*{Implementational details of the simulator}
Most of the simulator has been written for you but the details might be useful in analysing your results. The only parts of the simulator you will need to write are the different strategies used to add more arms to the contextual bandit. 

\noindent
The simulator is initialized with $N_u$ users and $N_a$ arms where each user and arm is represented as a feature vector of dimension $d$. Each element of these vectors is initialized i.i.d from a normal distribution. When reset, the simulator returns a random user context vector from the set of $N_u$ users.  When the algorithm chooses an arm, the simulator returns a reward of 0 if the arm chosen was the best arm for that user and -1 otherwise.

\noindent
We will be running the simulator for $T$ steps where each step represents one user interaction. After every $K$ steps, we add an arm to the simulator using one of three different strategies outlined below.

\noindent
Go through the code for the simulator in \texttt{submission.py}. Most of the simulator is implemented for you: the only method you will need to implement is \texttt{update\_arms()}. 

\begin{enumerate}[(a)]
    \subsection*{Implementing the Bandit Algorithm}

\item \points{2a}

For this assignment we will be using the Disjoint Linear Upper Confidence Bound (LinUCB) algorithm from \cite{li2010contextual}. 
The hints provided in \texttt{submission.py} should help you with this.
You have already implemented this in the previous problem but you will have to now fill the code in the 
\texttt{add\_arm\_params()} method to account for the fact that the number of arms could keep increasing now.


    \subsection*{Implementing the arm update strategies}

We will now implement three different strategies to add arms to our simulator. Each arm is associated with its true feature vector $\theta^*_a$. This is the $d$-dimensional feature vector we assigned to each arm when we initialized the simulator. This is the $\theta$ the LinUCB algorithm is trying to learn for each arm through $A$ and $b$. When we create new arms, we need to create these new feature vectors as well.

\noindent
When coming up with strategies to add arms, we need to put ourselves in the shoes of content creators and think about how we want to optimize for the videos that go up on our channels. When making such decisions, we only consider the previous $K$ steps since we added the last arm. Consider the three strategies outlined below:

\begin{enumerate}[(1)]
\item
\textbf{Popular:} For the last $K$ steps, pick the two most popular arms and create a new arm with the mean of the true feature vectors of these two arms. For example, assume $a_1$ and $a_2$ were the two most chosen arms in the previous $K$ steps with true feature vectors $\theta_1^*$ and $\theta^*_2$ respectively. Now create a new arm $a$ with $\theta_a^* = \frac{\theta_1^* + \theta_2^*}{2}$.\\
In the real world, this is similar to a naive approach where content creators create a new video based on their two most recommended videos from the last month.
\item
\textbf{Corrective:} For the last $K$ steps, consider all the users for whom we made incorrect recommendations. Assume we know what the best arm would have been for each of those users. Consider taking corrective action by creating a new arm that is the weighted mean of all these true best arms for these users. For example, say for the last $K$ steps, we got $n_1 + n_2$ predictions wrong where the true best arm was $a_1$ $n_1$ times and $a_2$ $n_2$ times. Create a new arm $a$ with $\theta_a^* = \frac{n_1\theta_1^* + n_2\theta_2^*}{n_1 + n_2}$. \\
In the real world, this is analogous to content creators adapting their content to give their viewers what they want to watch based on feedback from viewers about their preferences.

\item \textbf{Counterfactual:} Consider the following counterfactual: For the previous $K$ steps, had there existed an additional arm $a$, what would its true feature vector $\theta_a^*$ have to be so that it would have been better than the chosen arm at each of those $K$ steps? There are several ways to pose this optimization problem. Consider the following formulation:
\begin{align}
    \theta_a^* = \argmax_{\theta} \frac{1}{2}\sum_{i=1}^{K}(\theta^Tx_i - \theta_i^Tx_i)^2
\end{align}
Here $x_i$ is the context vector of the user at step $i$. $\theta_i$ is the true feature vector of the arm chosen at step $i$. We can now optimize this objective using batch gradient ascent.\\
\begin{align}
    \theta_a \leftarrow \theta_a + \eta \frac{\partial L}{\partial \theta}
\end{align}
Here $\eta$ is the learning rate and $L = \sum_{i=1}^{K}(\theta^Tx_i - \theta_i^Tx_i)^2$.\\
We can find $\frac{\partial L}{\partial \theta}$ directly as $\sum_{i=1}^{K} (\theta^Tx_i - \theta_i^Tx_i)x_i$. We can write the update rule as \\
\begin{align}
    \theta_a \leftarrow \theta_a + \eta \sum_{i=1}^{K} (\theta_a^Tx_i - \theta_i^Tx_i)x_i
\end{align}
In the real world, this is akin to asking the question, ``What item could I have recommended in the past $K$ steps that would have been better than all recommendations made in the past $K$ steps?'' In asking this, the creator aims to produce a new video that would appeal to all users more than the video that was recommended to them.
\end{enumerate}

\item \points{2b} Implement these three methods in the \texttt{update\_arms()} function in \texttt{submission.py}. This should be about 25 lines of code. Hints have been provided in the form of comments.

    \input{02-multi-arm-bandit/03-analysis}
    \input{02-multi-arm-bandit/04-discussion}
\end{enumerate}