\points{4b} \textbf{DDIM as Markovian Process}

We wish to introduce a Markovian variant of DDIM, we can incorporate a noise term at each step to make it probabilistic. 
Consider a parameter $\sigma_t$ that controls the amount of noise injected back into the process. By doing so, we can smoothly transition 
from a DDIM-like deterministic sampler (when $\sigma_t=0$) to a full stochastic, Markovian process (when $\sigma_t>0$).


To make the previous step Markovian, we introduce a noise term with variance controlled by $\sigma_t$, which is described by the formula (16) 
in \href{https://arxiv.org/pdf/2010.02502}{the DDIM paper}. Let $z \sim \mathcal{N}(0,I)$ be a standard Gaussian noise sample. 
The Markovian variant of the DDIM update can be written as:

\[
x_{t-1} =
\sqrt{\bar{\alpha}_{t-1}} x_0
+ 
\underbrace{\sqrt{1 - \bar{\alpha}_{t-1} - \sigma_t^2}\,\epsilon}_{\text{\texttt{predict\_sample\_direction}}} 
+ 
\underbrace{\sigma_t z}_{\text{\texttt{stochasticity\_term}}}.
\]

Please implement the following functions from the :
\begin{itemize}
    \item \texttt{get\_stochasticity\_std}
    \item \texttt{predict\_sample\_direction}
    \item \texttt{stochasticity\_term}
\end{itemize}

to accurately reproduce the update method described in the \href{https://arxiv.org/pdf/2010.02502}{the DDIM paper} by the formulas (12) and (16).

Run the experiment to generate few samples with \textbf{eta}=0, 0.2, 0.5, 0.75, 1 for \textbf{num\_steps} = 10 and provide a comment.